\documentclass{article}

\usepackage{amsmath}
\usepackage{cancel}
\begin{document}

\title{Derivation of Maxwell's theory of Conductivity}
\author{Craig Moir}
\maketitle

\section{For a single sphere}
Consider a single spherical particle immersed in an infinite fluid
where conduction is the only method of heat transport. A uniform heat
flux is imposed upon the system along the z-direction as shown in
Fig.~\ref{fig::system_diagram}. The equation for the temperature
distribution in the system at steady-state and with no generation or
consumption of heat is given by the Laplace equation,
\begin{align}
  \label{eq::laplace}
  \nabla^{2} T = 0,
\end{align}
where T is the temperature. The Laplace equation can be written in
spherical coordinates as
\begin{align}
  \label{eq::laplace_spherical_expand}
  \frac{1}{r^{2}} \frac{\partial}{\partial r} \left( r^{2} \frac{\partial}{\partial r} T \right) +
  \frac{1}{r^{2} \text{sin} \, \theta} \frac{\partial}{\partial \theta} \left(\text{sin} \, \theta \frac{\partial}{\partial r} T \right) +
  \frac{1}{r^{2} \text{sin}^{2} \theta} \frac{\partial^{2}}{\partial \phi^{2}} T
  = 0,
\end{align}
where r is the distance from the point of origin, $\theta$ is the
polar angle measured from the positive z-axis in the y-z plane, and
$\phi$ is the azimuthal angle measured from the positive x-axis in the
x-y plane. Since the heat flux is uniform along the z-direction, the
system will be symmetric in the azimuthal direction, $\partial T /
\partial \phi = 0$. Therefore, Eq.~\ref{eq::laplace_spherical_expand}
can be simplified as
\begin{align}
  \label{eq::laplace_spherical_expand_simplified}
  \frac{1}{r^{2}} \frac{\partial}{\partial r} \left( r^{2} \frac{\partial}{\partial r} T \right) +
  \frac{1}{r^{2} \text{sin} \, \theta} \frac{\partial}{\partial \theta} \left(\text{sin} \, \theta \frac{\partial}{\partial r} T \right)
  = 0.
\end{align}

This equation has multiple solutions such as $T(r, \theta)=1$,
$T(r, \theta) = r \, \text{cos} \, \theta$, and
$T(r, \theta) = \text{cos} \, \theta / r^{2}$ which we confirm by
substitution into the equation. Assuming this is all of the relevant
solutions, a general solution can be made up of a sum of these terms:
\begin{align}
  %\label{eq::laplace_sol_out}
  T_{\alpha} &= A_\alpha \, r\,\cos \theta +  B_\alpha\frac{\cos{\theta}}{r^{2}}  + C_\alpha + \ldots
\end{align}

For the outer part, we know that as $r\to\infty$, $T_{out}\to = T_0 +
\left(\frac{{\rm d}T}{{\rm d} z}\right)_{\infty}\,z$. Taking the same
limit of the equation we see this functional form arise,
\begin{align}
  %\label{eq::laplace_sol_out}
  \lim_{r\to\infty}T_{out} &= A_{out} \, r\,\cos \theta +  \cancelto{0}{B_{out}\frac{\cos{\theta}}{r^{2}}}  + C_{out}
  \\
  &= A_{out} \, z + C_{out} &
\end{align}
Therefore $C_{out}=T_0$ and $A_{out}=\left(\frac{{\rm d}T}{{\rm d}
  z}\right)_{\infty}$.
Giving 
\begin{align}
  %\label{eq::laplace_sol_out}
  T_{out} &= \left(\frac{{\rm d}T}{{\rm d}
    z}\right)_{\infty} r\,\cos \theta +  B_{out}\frac{\cos{\theta}}{r^{2}}  + T_0 + \ldots
\end{align}
where $z=r\,\cos\theta$. Comparing this to the limiting form we can
determine that $C_{out}=T_0$ and
$A_{out}=\left(\frac{{\rm d}T}{{\rm d} z}\right)_{\infty}$; however, $B_{out}$ remains undetermined,
\begin{align}
  %\label{eq::laplace_sol_out}
  T_{out} &=\left(\frac{{\rm d}T}{{\rm d}
    z}\right)_{\infty} r\,\cos \theta +  B_{out}\frac{\cos{\theta}}{r^{2}}  + T_0
\end{align}
Considering the inner section now,
\begin{align}
  %\label{eq::laplace_sol_out}
  T_{in} &= A_{in} \, r\,\cos \theta +  B_{in}\frac{\cos{\theta}}{r^{2}}  + C_{in}
\end{align}
We know that as $r\to0$, $T$ must remain finite, therefore $B_{in}=0$.
\begin{align}
  T_{in} &= A_{in} \, r\,\cos \theta  + C_{in}
\end{align}
The remaing boundary conditions of the system are
\begin{align}
  T_{in} &= T_{out} & \text{for $r=R$} \label{eq:noslipT}
  \\
  %\label{bc::surface_flux}
  - k_{in} \frac{\partial T_{in}}{\partial r} &= - k_{out} \frac{\partial T_{out}}{\partial r}, & \text{for $r=R$}
\end{align}
where $k_{in}$, and $k_{out}$ denote the thermal conductivity of the
material inside and outside the sphere
respectively. Eq.~\ref{eq:noslipT} assumes that there is no
interfacial resistance between the sphere and the fluid. Using this
first boundary condition and applying it we have the following,
\begin{align}
  \left(\left(\frac{{\rm d}T}{{\rm d}
  z}\right)_{\infty} R +  \frac{B_{out}}{R^{2}}\right)\cos \theta  + T_0 &= A_{in} \, R\,\cos \theta  + C_{in}
\end{align}
As $\theta$ can vary independently, the equality is only satisfied if
$C_{in}=T_0$, and the $\cos\theta$ terms are equal,
\begin{align}\label{eq:costhetaterms}
  A_{in} &= \left(\frac{{\rm d}T}{{\rm d}
  z}\right)_{\infty} +  \frac{B_{out}}{R^{3}}
\end{align}
Consider the second boundary condition, the radial heat flux is
$q_r=-k\,\partial T/\partial r$ and equating the inner and outer
fluxes at $r=R$ yields the following,
\begin{align}
  k_{in} A_{in} \cos \theta &= k_{out}\left(\left(\frac{{\rm d}T}{{\rm d}
                              z}\right)_{\infty} \cos \theta -2\,R^{-3}\,B_{out}\cos{\theta}\right)
  \\
   A_{in} &= \frac{k_{out}}{k_{in}}\left(\left(\frac{{\rm d}T}{{\rm d}
                  z}\right)_{\infty} -2\,R^{-3}\,B_{out}\right)
\end{align}
Substituting equating this with Eq.~\ref{eq:costhetaterms},
\begin{align}
\frac{k_{out}}{k_{in}}\left(\left(\frac{{\rm d}T}{{\rm d}
                  z}\right)_{\infty} -2\,R^{-3}\,B_{out}\right) &= \left(\frac{{\rm d}T}{{\rm d}
                                                                  z}\right)_{\infty} +  \frac{B_{out}}{R^{3}}
  \\
  B_{out} = R^3\frac{\frac{k_{out}}{k_{in}}-1}{1+2\frac{k_{out}}{k_{in}}}\left(\frac{{\rm d}T}{{\rm d}
                  z}\right)_{\infty}
\end{align}
Which implies,
\begin{align}
  A_{in} &= \frac{3\frac{k_{out}}{k_{in}}}{1+2\frac{k_{out}}{k_{in}}}\left(\frac{{\rm d}T}{{\rm d} z}\right)_{\infty}
\end{align}
Putting this all together,
\begin{align}
  T_{in} &= z\frac{3\frac{k_{out}}{k_{in}}}{1+2\frac{k_{out}}{k_{in}}}\left(\frac{{\rm d}T}{{\rm d} z}\right)_{\infty}+ T_0
  \\
  T_{out} &=z\left(1 +  \frac{R^3}{r^{3}}\frac{\frac{k_{out}}{k_{in}}-1}{1+2\frac{k_{out}}{k_{in}}}\right)\left(\frac{{\rm d}T}{{\rm d}
                  z}\right)_{\infty}  + T_0
\end{align}
This is the solution for temperature around a sphere embedded in a
linear temperature gradient. We note, that the temperature change
caused by a single sphere over the background linear temperature profile is
\begin{align*}
\Delta T_{out} &=z\frac{R^3}{r^{3}}\frac{\frac{k_{out}}{k_{in}}-1}{1+2\frac{k_{out}}{k_{in}}}\left(\frac{{\rm d}T}{{\rm d}z}\right)_{\infty}
\end{align*}
\section{For a mixture}
Now consider a mixture of spheres. Assume the spheres are far apart,
so that their effect on the temperature profile is just the sum of
their changes from the basic linear temperature profile:

\begin{align}
  T_{out}^{(mix)}(r) &=z\left(1+\sum_{i}^{N_p}\frac{R_{nano}^3}{r^{3}_i}\frac{\frac{k_{out}}{k_{in}}-1}{1+2\frac{k_{out}}{k_{in}}}\right)\left(\frac{{\rm d}T}{{\rm d}
                    z}\right)_{\infty}  + T_0
  \\
 &\approx z\left(1+N_p\frac{R_{nano}^3}{r^{3}}\frac{\frac{k_{out}}{k_{in}}-1}{1+2\frac{k_{out}}{k_{in}}}\right)\left(\frac{{\rm d}T}{{\rm d}
                  z}\right)_{\infty}  + T_0
\end{align}
where in the second line we assume that we are so far away from the
nanoparticles, that they are collected near to the origin and so the
sum is replaced with a multiplication. Now consider a sphere around
the origin and nanoparticles whose radius is such that it has some
particular volume fraction,
\begin{align*}
  \phi &= \frac{V_{nano}}{V_{sys}}=\frac{N_p\,4\,\pi\,R^3_{nano}/3}{4\,\pi\,R^3_{sys}/3}
  \\
  R^3_{sys}&= \frac{N_p\,R^3_{nano}}{\phi}
\end{align*}
What is the effective thermal conductivity, $k_{eff}$ of this sphere
of radius $R_{sys}^3$ which gives the same temperature profile as the
multi-particle system? I.e., when is the following statement true
\begin{align*}
  T_{out}^{(mix)}(r) &=z\left(1+N_p\frac{R_{nano}^3}{r^{3}}\frac{\frac{k_{out}}{k_{in}}-1}{1+2\frac{k_{out}}{k_{in}}}\right)\left(\frac{{\rm d}T}{{\rm d}z}\right)_{\infty}  + T_0
  \\
  &=z\left(1+\frac{R_{sys}^3}{r^{3}}\frac{\frac{k_{out}}{k_{eff}}-1}{1+2\frac{k_{out}}{k_{eff}}}\right)\left(\frac{{\rm d}T}{{\rm d}z}\right)_{\infty}  + T_0
\end{align*}
Eliminating common terms and subsituting in the expression for
$R^3_{sys}$ we have,
\begin{align*}
  N_p\frac{R_{nano}^3}{r^{3}}\frac{\frac{k_{out}}{k_{in}}-1}{1+2\frac{k_{out}}{k_{in}}} = N_p\frac{R^3_{nano}}{r^{3}}\frac{1}{\phi}\frac{\frac{k_{out}}{k_{eff}}-1}{1+2\frac{k_{out}}{k_{eff}}}
\end{align*}
Cancelling,
\begin{align*}
\phi\frac{\frac{k_{out}}{k_{in}}-1}{1+2\frac{k_{out}}{k_{in}}} = \frac{\frac{k_{out}}{k_{eff}}-1}{1+2\frac{k_{out}}{k_{eff}}}
\end{align*}
Rearranging for $k_{eff}$,
\begin{align*}
  \phi\frac{\frac{k_{out}}{k_{in}}-1}{1+2\frac{k_{out}}{k_{in}}} &= \frac{k_{out}-k_{eff}}{k_{eff}+2\,k_{out}}
  \\
  (k_{eff}+2\,k_{out})\phi\frac{\frac{k_{out}}{k_{in}}-1}{1+2\frac{k_{out}}{k_{in}}} &= k_{out}-k_{eff}
  \\
  (k_{eff}+2\,k_{out})\phi\frac{1-\frac{k_{out}}{k_{in}}}{1+2\frac{k_{out}}{k_{in}}} &= k_{eff}-k_{out}
  \\
  k_{out}\left(2\,\phi\frac{1-\frac{k_{out}}{k_{in}}}{1+2\frac{k_{out}}{k_{in}}}+1\right) &= k_{eff}\left(1-\phi\frac{1-\frac{k_{out}}{k_{in}}}{1+2\frac{k_{out}}{k_{in}}}\right)
  \\
  k_{out}\left(2\,\phi\frac{1-\frac{k_{out}}{k_{in}}}{1+2\frac{k_{out}}{k_{in}}}+1\right) &= k_{eff}\left(1+\phi\frac{\frac{k_{out}}{k_{in}}-1}{1+2\frac{k_{out}}{k_{in}}}\right)
  \\
  k_{out}\left(2\,\phi\left(1-\frac{k_{out}}{k_{in}}\right)+1+2\frac{k_{out}}{k_{in}}\right) &= k_{eff}\left(\phi\frac{k_{out}}{k_{in}}-1+1+2\frac{k_{out}}{k_{in}}\right)
  \\
  k_{out}\left(2\,\phi+1+2(1-\phi)\frac{k_{out}}{k_{in}}\right) &= k_{eff}\left(\phi+2\right)\frac{k_{out}}{k_{in}}
  \\
  \frac{k_{eff}}{k_{out}}&=\frac{\frac{k_{in}}{k_{out}}(2\,\phi+1)+2(1-\phi)}{\phi+2}
\end{align*}
\end{document}
